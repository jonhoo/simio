\documentclass{scrartcl}
\usepackage{hyperref}

\title{6.852 Project Status Report}
\subtitle{A Visual Simulator for I/O Automata}
\author{Jon Gjengset \and Cody Cutler}

\begin{document}
\maketitle

\section{Project description}

We aim to build an simulator that can simulate the behaviour of an algorithm
specified using I/O automata when run on an arbitrary network graph. In
particular, we want the user of the system to be able to write a specification
of the algorithm, as well as designate some particular network graph, and then
be able to simulate and visualize the algorithm's behaviour when run on all the
nodes in the network.

\section{Related work}

Of particular note in terms of related work are the
\href{http://groups.csail.mit.edu/tds/ioa/}{IOA Language and Toolset}, which
simulates the behaviour of a single I/O automata, and the
\href{http://www.veromodo.com/}{Tempo Toolset}, which simulates a single
*timed* I/O automata. These tools provide the user with a domain-specific
language for describing automata, as well as tools for simulating the operation
of a single automata, doing invariant checking, verifying liveness and safety
properties, and generating code.

While these both provide authors of automata with powerful tools for verifying
the correctness of their algorithms, they are not particularly useful for
users who want to study the behaviour of an automata so as to better understand
the underlying algorithm. In particular, we envision that students who are
being introduced to reasoning about distributed systems using I/O automata for
the first time, or seasoned researchers toying with an idea for a new
algorithm, would benefit greatly from being able to see how a network of nodes
behaves when running a specific algorithm. Furthermore, we believe that, by
allowing the user to specify the network graph they are interested in, they can
quickly explore corner cases that they suspect might be difficult for the
algorithm, or they can improve their understanding of the algorithm by
observing its behaviour in very different network topologies.

Another reason why we feel existing solutions are insufficient is that they
require the user to learn a new programming language simply to be able to type
up an algorithm. Instead, we propose to provide the user with the ability to
define their automata using the high-level programming language Python, which
many computer scientists are already familiar with. State variables will be
represented as Python variables, giving the user access to the suite of
primitive types and classes provided by Python and its many third-party
modules, without having to implement all these types ourselves. In the same
vein, transitions will be specified as pure Python functions whose arguments
are the current state variables, as well as any transition inputs, and whose
return value will be the new state of the object. Modeling transitions this way
allows us to give the full power of the Python language to the user, without
incurring additional implementation cost for us.

\section{Challenges}

The flexibility of expression in our system means that doing formal verification
or analysis of the automata becomes difficult. We surmise that this is the
reason why previous work generally chose to implement a language from scratch;
it gives them the opportunity to limit the language to only provide features
that can be reasoned about using current program analysis techniques. We
therefore position our simulator as a complement to existing tools, rather than
as a replacement for them. In particular, we anticipate that users will first
prototype their automata in our simulator to examine their behaviour, and then,
once the algorithm has been sufficiently well refined, write their final
automata specification using a specialized language such as Tempo so that
formal methods can be used to prove properties of the final algorithm.

\section{Project status}

At this stage in the project's development, we have built rudimentary support
for specifying automata, as well as a visualizing front-end that can display
nodes in a network, as well as show communication between nodes. While these
will both likely need further refinement, they are at the stage where we can
now focus on building a working automata simulator. This simulator would first
invoke the parser to get an executable version of the automata, construct and
initialize several copies of the automata to correspond to each of the nodes in
the simulated network, and link each automata as dictated by the network graph
specification. Once this setup has been performed, it would then start
executing the automata concurrently, ensuring fairness to each automata's task.
The information about messages being sent and nodes changing state would then
be communicated to the rendered in real-time, so that the visualization of the
running algorithm can be produced.

We anticipate that the biggest hurdle to making the simulator will be finding a
good way of handling non-determinism in the simulated automata while still
providing fairness. The reason for this is that a user may wish to bias the
non-determistic choices to force the algorithm to behave in a particular way.
It is not obvious how this can be done in a way that is both easy for the user,
and, at the same time, does not get in the way of the simulator ensuring that
the overall execution remains fair.
\end{document}
