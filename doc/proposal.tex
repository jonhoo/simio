\documentclass{scrartcl}
\usepackage{hyperref}

\title{6.852 Project Proposal}
\subtitle{A Visual Simulator for I/O Automata}
\author{Jon Gjengset \and Srivatsa S. Bhat \and Cody Cutler}

\begin{document}
\maketitle

We propose to develop a tool to visualize the behavior of I/O Automata in
Asynchronous Distributed Systems.

The I/O Automata simulator will be able to accept specifications in the style
of the pseudo-code in the book. The tool would also allow the user to connect
the given automata in arbitrary network topologies, control its inputs, and
vary other interesting properties like UID configuration or link delay.

The simulations would be shown by visualizing the system topology and
displaying the actions taken by each automaton as the components interact with
each other at every step.

In addition to this, the tool would also potentially be able to output the
empirical (lower/upper) bounds on the time and communication complexity of the
specified distributed algorithm, by actually evaluating them for the particular
scenarios determined by the user-inputs.  This will help us observe the
theoretical bounds in practice, by choosing our inputs carefully (such that
they bring about the best/worst-case behavior of the algorithm). 

We believe that such a tool would be helpful for students to grasp complex
distributed algorithms easily, since it provides a platform to visualize the
various scenarios and also control and observe the behavior of the I/O automata
under varying circumstances.

\vspace{2mm}
\textbf{References:}

1. Nancy Lynch, \textit{Distributed Algorithms.} Morgan Kaufmann Publishers, 1996.

2. Hagit Attiya and Jennifer Welch, \textit{Distributed Computing: Fundamentals, Simulations, and Advanced Topics.} John Wiley and Sons, Inc., 2004. Second Edition
\end{document}
