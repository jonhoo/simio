\documentclass{scrartcl}
\usepackage[usenames,dvipsnames]{xcolor}
\usepackage{caption, subcaption}

\usepackage{framed}
\usepackage{svg}
\setsvg{svgpath = figures/}

\def \kw#1{\texttt{\color{MidnightBlue}#1}}
\def \self {\texttt{\color{Cerulean}self}}
\def \py#1{\texttt{\color{LimeGreen}#1}}
\def \pyval#1{\texttt{\color{Sepia}#1}}

\title{6.852 Project Report}
\subtitle{A Visual Simulator for I/O Automata}
\author{Jon Gjengset \and Cody Cutler}

\begin{document}

\maketitle
\begin{abstract}
	Designing and implementing correct distributed, asynchronous algorithms
	is hard, and understanding an algorithm designed by other people
	especially so. Newcomers to the field face unfamiliar concepts and
	terminology when being introduced to these kinds of algorithms, and
	deriving intuition about why these algorithms work solely from their
	algorithmic descriptions can be far from trivial. We present Simio; a
	simulator for I/O automata that visualize the behaviour of distributed
	algorithms in network graphs specified by the user. The automata are
	specified using the popular Python programming language, making it easy
	for students of distributed algorithms to code up and simulate new
	algorithms. We believe this tool will aid in developing understanding
	about the behaviour of algorithms, as well as serve as a quick
	prototyping tool to test whether a propsed algorithm may solve some
	problem.
\end{abstract}

\section{Introduction}
Distributed algorithms are frequently described in terms of a particular I/O
automaton that defines the behaviour of some set of nodes in a system. These
automata are used heavily in distributed algorithm theory to reason about and
prove properties of various algorithms, and can be used in a variety of
systems.

An I/O automaton specifies a set of state variables for each node in the
system, as well as a series of transitions with preconditions and effects that
take a node from one state to another, and one instance of such an atomaton can
be seen much as a state machine. When multiple such automata instances are
connected (e.g.\ through a network or distibuted shared memory), the output of
one automaton can induce state transitions on another automaton. This is what
enables I/O automata to encapsulate a distributed algorithm.

Intuitively understanding the behaviour of an algorithm when specified through
an I/O automaton alone can be difficult, as it is the combined behaviour of
many nodes that the reader is usually interested in. While several simulators
exist that simulate a single automaton (usually by having it interact directly
with an ``environment automaton'' specified by the user), these are not
directly useful for observing the overall behaviour of the algorithm.

While this kind of simulation is sufficient for individuals familiar with the
algorithms in question who want to verify some aspect of the automaton's
behaviour, it is of less use to users who wish to understand some new,
unfamiliar algorithm. For this project, we developed Simio, an I/O automaton
simulator that simulates \textbf{and visualizes} the behaviour of a
user-specified I/O automaton in a network. We hope this tool can be used by
students of the field to gain a better understanding of the overall behaviour
of various algorithms, and explore how they work in different scenarios. The
following report details the specification langauge for the automata, the
design of the simulator, its limitations, and some observations about its uses.

\section{I/O automaton specifications}
We wanted our I/O automaton specifications to resemble the style of the
pseudo-code that I/O automata are usually given in literature. In particular,
we based our design on the format used in the well-known Nancy Lynch's
Distributed Algorithm book\cite{book}. Contrary to other, exisisting simulators
that develop a new language entirely for specifying these automata so that
formal verification of their behaviour can be performed, we opted to sacrifice
formalism for flexibility. Our specifications are given in Python, with some
added syntactic constructs for communicating the role of different code
segments to the simulator. This gives users the ability to use all existing
Python libraries and data-types without requiring us to do extra work to make
them work.

To illustrate, we give the specification for the well-known Bellman-Ford
algorithm for computing the shortest path from a particular node to every other
node in a graph in Figure~\ref{alg:1}. The keywords highlighted in \kw{blue}
are recognized by our parser, and are used to denote the different sections of
an automaton. Text inside the \kw{State}, \kw{Transition-Precondition},
\kw{Transition-Output} and \kw{Transition-Effect} blocks are all pure Python
code.

The code in these blocks are given access to the current instance of the
automaton (i.e.\ its current state) through the \self\ variable, and are
expected to return a value appropriate to the containing block. For example,
the code in a \kw{Transition-Precondition} block is expected to return true
when the transition is enabled, and false otherwise. There are also certain
special variables such as \kw{self.i}; the node's unique identifier, \kw{to};
the destination node being considered for an output transition, and \kw{fr};
the sender of a message received by an input transition.

Of particular note is the use of the \kw{connect} keyword in the
\kw{Signature-Output} block. This is a hint to the simulator that the outputs
from this particular transition should be fed into the indicated input
transition of a neighbouring node in the simulated network.

\begin{figure}
\begin{framed}
\begingroup
\parindent=0pt\relax
\setlength{\parskip}{2pt}
\footnotesize
\ttfamily
\obeyspaces%
\let =\ %
\obeylines%
\kw{Name}:
    BellmanFord

\kw{Signature-Input}:
    receive(m)

\kw{Signature-Output}:
    send(m) \kw{connect} receive

\kw{State}:
    \self.parent = \pyval{None}
    \self.updated = []
    \self.dist = \pyval{None}
    \py{if} \kw{self.i} == \pyval{0}:
        \self.dist = \pyval{0}

\kw{Transition-Name}:
    send(m)
\kw{Transition-Precondition}:
    \py{return} \self.dist \py{is} \py{not} \pyval{None} \py{and} \kw{to} \py{not} \py{in} \self.updated
\kw{Transition-Output}:
    \py{return} \self.dist
\kw{Transition-Effect}:
    \self.updated.append(\kw{to})

\kw{Transition-Name}:
    receive(m)
\kw{Transition-Effect}:
    act = m + \self.weights[\kw{fr}]
    \py{if} \self.dist \py{is} \pyval{None} \py{or} act < \self.dist:
        \self.dist = act
	\self.parent = \kw{fr}
        \self.updated = []

\kw{Tasks}:
    send(m)
\endgroup
\end{framed}
\caption{Simio description of Bellman-Ford}\label{alg:1}
\end{figure}

We note that the code reads very similar to a formal I/O automaton
specification, which is one of Simio's goals. Simio now takes such an automaton
specification and parses it into an executable Python module, which is then
used by the simulator to run multiple nodes connected in a network.

\section{Simulation}\label{sec:sim}

The simulator takes as input both the output of the parser and a description of
a (directional) graph. This input is then used to execute a simulation as
follows. First, the simulator builds the simulation network, where each node in
the input graph represents an automaton of the type specified in the parser
output. Each of these nodes is then given information that is usually assumed
to be known to each node under the asynchronous I/O automata model; this
includes the number of nodes in the graph, the set of neighbouring nodes,
weights of adjancent edges, and per-node unique identifiers.

Next, channels are inserted between each pair of nodes that are connected by an
edge in the graph. Channels are also represented internally by a Simio
specification and are thus completely configurable. The channels are subject to
the same scheduling (discussed below) as the user-defined automata. The
simulator then repeatedly executes exactly one enabled action at a time until
no more enabled actions remain.

The way in which the simulator chooses the next action to run (the
``scheduler'') is an important part of the simulator, and warrants some
explanation. As a strawman example, consider a naive scheduler that simply
executes automaton actions in round-robin order; every step, it will consider
only the enabled transitions of node $n+1$. This effectively forces the nodes
to advance in lock-step with each other, and will fail to ever simulate
executions in which one node takes multiple steps while another node takes
none. Since these are the interesting cases for many distributed algorithms
(the problem otherwise reduces to something akin to the synchronous setting),
this scheduler is not particularly useful.

The asynchronous I/O automaton model places no constraints on how often the
automata execute transitions relative to each other. For Simio, we therefore
decided to make our scheduler random --- any enabled action of any automaton
may executed at any time. This design choice was made for two reasons:
First, given a decent random number generator, a random scheduler is guaranteed
to be a fair scheduler; each node will take a step with some non-zero
probability in each round, meaning it must eventually take a step as the number
of steps tend to infinity. This argument extends to the statement that any
enabled action must occur infinitely often, which is sufficient to show
fairness.

Second, executing actions in a random order preserves the uncertainty that is
inherent in the asynchronous I/O automaton model. Realistically, nodes run at
different speeds, and common delays such as link latency and CPU scheduling
affect the relative ordering of events at different nodes, a fact which robust
distrbuted algorithms must take into account. These effects are approximated by
a random scheduler, since actions can now be interleaved arbitrarily. We
believe the random scheduler is thus a good fit for a simulator, since it
exposes realistic behavior to the student, and precisely because no two
executions will be the same.

\section{Related work}

I/O automata are used widely in the distributed systems theory community, and
several simulators exist. Of particular note is the IOA Language and
Toolset\cite{ioa}, which simulates the behaviour of a single I/O automaton, and
the Tempo Toolkit\cite{tempo}, which simulates a single \textit{timed} I/O
automaton. These tools provide the user with a domain-specific language for
describing automata, as well as tools for simulating the operation of a single
automaton, doing invariant checking, verifying liveness and safety properties,
and generating code.

While these both provide authors with powerful tools for verifying the
correctness of their algorithms, they are not particularly useful for users who
want to study the behaviour of an I/O automaton so as to better understand the
underlying algorithm. As mentioned previously, we envision that students who
are being introduced to the field of distributed systems theory and I/O
automata for the first time, or seasoned researchers prototyping an idea for a
new algorithm, would benefit greatly from being able to \textit{see} how a
network of nodes behave when running some algorithm. Furthermore, we believe
that, by allowing the user to specify the network graph they are interested in,
they can quickly explore corner cases that they suspect might be difficult for
the algorithm, or they can improve their understanding of the algorithm by
observing its behaviour in very different network topologies.

Another reason why we feel existing solutions are insufficient is that they
require the user to learn a new programming language simply to be able to type
up an algorithm. Instead, Simio provides the user with the ability to define
their automata using the high-level programming language Python, which many
computer scientists are already familiar with. This gives users access to the
suite of primitive types, classes and functions provided by Python and its many
third-party modules, without forcing us to re-develop this infrastructure
ourselves.

\section{Limitations}

The flexibility of expression in our system means that doing formal verification
or analysis of the automata becomes difficult. We surmise that this is the
reason why previous work generally chose to implement a language from scratch;
it gives them the opportunity to limit the language to only provide features
that can be reasoned about using current program analysis techniques. We
therefore position our simulator as a supplement to existing tools, rather than
as a replacement for them. In particular, we anticipate that users will first
prototype their automata in our simulator to examine their behaviour, and then,
once the algorithm has been sufficiently well refined, write their final
automata specification using a specialized language such as Tempo, so that
formal methods can be used to prove properties of the final algorithm.

As described in Section~\ref{sec:sim}, Simio currently only supports the
scenario where all the nodes in the network are running instances of the same
I/O automaton. This is not an inherent limitation of I/O automata, but rather a
choice we have made to simplify implementation. Our choice not to support this
feature effectively prohibits users from composing automata, and potentially
forces them to specify larger and more complex automata to emulate this kind of
behaviour in our system.

Simio, at the time of writing, is also limited to simulating nodes in networks.
Distributed shared memory systems are very similar, and open up the world of
visualized simulations to a whole new class of algorithms, but unfortunately,
time constraints prevent us from supporting this in the current version of
Simio. Networks where nodes fail or messages are lost also fall into this
category, as they are effectively different systems in which algorithms can
run. We are leaving such alternative system models as future work.

A final limitation that is also an intentional design choice, is that the user
is not able to specify the scheduling of messages and transitions. Instead, we
have chosen, as discussed above, to completely randomize the scheduling
behaviour of the simulator. While we may add this feature in the future, we
believe it is more important to expose the user to many \textit{different} runs
of the algorithm, than it is to let them force a particular behaviour. Simio is
built to explore and develop an understanding for algorithms, not as an
exhaustive algorithm testing tool.

\section{Visualizations}

Simio uses the popular Graphviz library\cite{gv} to render the nodes and edges
in the network, as well as to convey information about the current state of the
simulation. Figure~\ref{fig:viz} shows the rendered visualization at various
stages throughout the simulation of Bellman-Ford on an arbitrary graph.

\begin{figure}[h]
	\centering
	\begin{subfigure}[b]{0.49\textwidth}
		\includesvg[width=\textwidth]{start}
		\caption{Root broadcast}
	\end{subfigure}
	\begin{subfigure}[b]{0.49\textwidth}
		\includesvg[width=\textwidth]{one-hop}
		\caption{Node forwards}
	\end{subfigure}
	\\
	\begin{subfigure}[b]{0.49\textwidth}
		\includesvg[width=\textwidth]{far}
		\caption{Algorithm progresses}
	\end{subfigure}
	\begin{subfigure}[b]{0.49\textwidth}
		\includesvg[width=\textwidth]{end}
		\caption{Algorithm terminates}
	\end{subfigure}
	\caption{Steps of Bellman-Ford. Red arrows indicate parent pointers,
		blue arrows indicates messages being sent across channels. The
		full simulator also shows node identifiers and current best
		distance estimates, but these have been omitted for clarity.}\label{fig:viz}
\end{figure}

The visualization is arguably much more striking when seen live, so we invite
the initiated reader to download a copy of Simio and run the Bellman-Ford demo
using \texttt{examples/demos/bellmanford.sh} to see Simio in action.

\section{Code release}

The souce code for Simio has been released open-souce on GitHub\cite{simio}.
The release includes working I/O automata implementations of, as well as demos
for, the Bellman-Ford algorithm, Peterson's algorithm for leader-election in
unidirectional rings\cite{peterson}, the Paxos leader election
algorithm\cite{paxos}, and Michael Luby's algorithm for determining the Maximal
Independent Set of a graph\cite{lubymis}.

\begin{thebibliography}{1}
\bibitem{book} Nancy A. Lynch.\@ 1996.\@ {\em Distributed Algorithms}.\@ Morgan Kaufmann Publishers Inc., San Francisco, CA, USA. 
\bibitem{ioa} MIT Theory of Distributed Systems Group.\@ IOA Language and Toolset.\@ http://groups.csail.mit.edu/tds/ioa/
\bibitem{tempo} Veromodo.\@ The Tempo Toolkit.\@ http://www.veromodo.com/
\bibitem{gv} Graphviz.\@ Graph Visualization Software.\@ http://graphviz.org/
\bibitem{simio} GitHub.\@ Simio source code.\@ https://github.com/jonhoo/simio

\bibitem{paxos} Leslie Lamport.\@ 1998.\@ {\em The part-time parliament}.\@ ACM Trans.  Comput. Syst. 16, 2 (May 1998), 133-169.
\bibitem{peterson} Gary L. Peterson.\@ 1982.\@ {\em An O(nlog n) Unidirectional Algorithm for the Circular Extrema Problem}.\@ ACM Trans. Program. Lang.  Syst. 4, 4 (October 1982), 758-762.
\bibitem{lubymis} M Luby.\@ 1985.\@ {\em A simple parallel algorithm for the maximal independent set problem}.\@ In Proceedings of the seventeenth annual ACM symposium on Theory of computing (STOC '85). ACM, New York, NY, USA, 1-10.
\end{thebibliography}
\end{document}
