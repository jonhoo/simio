\documentclass{scrartcl}
\usepackage{framed}
\usepackage{hyperref}
\usepackage[usenames,dvipsnames]{xcolor}
\def \kw#1{\texttt{\color{MidnightBlue}#1}}
\def \self {\texttt{\color{Cerulean}self}}
\def \py#1{\texttt{\color{LimeGreen}#1}}
\def \pyval#1{\texttt{\color{Sepia}#1}}

\title{6.852 Project Report}
\subtitle{A Visual Simulator for I/O Automata}
\author{Jon Gjengset \and Cody Cutler}

\begin{document}

\maketitle
\begin{abstract}
	Designing and implementing correct distributed, asynchronous algorithms
	is hard, and understanding an algorithm designed by other people
	especially so. Newcomers to the field face unfamiliar concepts and
	terminology when being introduced to these kinds of algorithms, and
	deriving intuition about why these algorithms work solely from their
	algorithmic descriptions can be far from trivial. We present Simio; a
	simulator for I/O automata that visualize the behaviour of distributed
	algorithms in network graphs specified by the user. The automata are
	specified using the popular Python programming language, making it easy
	for students of distributed algorithms to code up and simulate new
	algorithms. We believe this tool will aid in developing understanding
	about the behaviour of algorithms, as well as serve as a quick
	prototyping tool to test whether a propsed algorithm may solve some
	problem.
\end{abstract}

\section{Introduction}
Distributed algorithms are frequently described in terms of a particular I/O
automaton that defines the behaviour of some set of nodes in a system. These
automata are used heavily in distributed algorithm theory to reason about and
prove properties of various algorithms, and can be used in a variety of
systems.

An I/O automaton specifies a set of state variables for each node in the
system, as well as a series of transitions with preconditions and effects that
take a node from one state to another, and one instance of such an atomaton can
be seen much as a state machine. When multiple such automata instances are
connected (e.g.\ through a network or distibuted shared memory), the output of
one automaton can induce state transitions on another automaton. This is what
enables I/O automata to encapsulate a distributed algorithm.

Intuitively understanding the behaviour of an algorithm when specified through
an I/O automaton alone can be difficult, as it is the combined behaviour of
many nodes that the reader is usually interested in. While several simulators
exist that simulate a single automaton (usually by having it interact directly
with an ``environment automaton'' specified by the user), these are not
directly useful for observing the overall behaviour of the algorithm.

While this kind of simulation is sufficient for individuals familiar with the
algorithms in question who want to verify some aspect of the automaton's
behaviour, it is of less use to users who wish to understand some new,
unfamiliar algorithm. For this project, we developed Simio, an I/O automaton
simulator that simulates \textbf{and visualizes} the behaviour of a
user-specified I/O automaton in a network. We hope this tool can be used by
students of the field to gain a better understanding of the overall behaviour
of various algorithms, and explore how they work in different scenarios. The
following report details the specification langauge for the automata, the
design of the simulator, its limitations, and some observations about its uses.

\section{I/O automaton specifications}
We wanted our I/O automaton specifications to resemble the style of the
pseudo-code that I/O automata are usually given in literature. In particular,
we based our design on the format used in the well-known Nancy Lynch's
Distributed Algorithm book. Contrary to other, exisisting simulators that
develop a new language entirely for specifying these automata so that formal
verification of their behaviour can be performed, we opted to sacrifice
formalism for flexibility. Our specifications are given in Python, with some
added syntactic constructs for communicating the role of different code
segments to the simulator. This gives users the ability to use all existing
Python libraries and data-types without requiring us to do extra work to make
them work.

To illustrate, we give the specification for the well-known Bellman-Ford
algorithm for computing the shortest path from a particular node to every other
node in a graph in Figure~\ref{alg:1}. The keywords highlighted in \kw{blue}
are recognized by our parser, and are used to denote the different sections of
an automaton. Text inside the \kw{State}, \kw{Transition-Precondition},
\kw{Transition-Output} and \kw{Transition-Effect} blocks are all pure Python
code.

The code in these blocks are given access to the current instance of the
automaton (i.e.\ its current state) through the \self\ variable, and are
expected to return a value appropriate to the containing block. For example,
the code in a \kw{Transition-Precondition} block is expected to return true
when the transition is enabled, and false otherwise. There are also certain
special variables such as \kw{self.i}; the node's unique identifier, \kw{to};
the destination node being considered for an output transition, and \kw{fr};
the sender of a message received by an input transition.

Of particular note is the use of the \kw{connect} keyword in the
\kw{Signature-Output} block. This is a hint to the simulator that the outputs
from this particular transition should be fed into the indicated input
transition of a neighbouring node in the simulated network.

\begin{figure}\label{alg:1}
\begin{framed}
\begingroup
\parindent=0pt\relax
\setlength{\parskip}{2pt}
\footnotesize
\ttfamily
\obeyspaces%
\let =\ %
\obeylines%
\kw{Name}:
    BellmanFord

\kw{Signature-Input}:
    receive(m)

\kw{Signature-Output}:
    send(m) \kw{connect} receive

\kw{State}:
    \self.parent = \pyval{None}
    \self.updated = []
    \self.dist = \pyval{None}
    \py{if} \kw{self.i} == \pyval{0}:
        \self.dist = \pyval{0}

\kw{Transition-Name}:
    send(m)
\kw{Transition-Precondition}:
    \py{return} \self.dist \py{is} \py{not} \pyval{None} \py{and} \kw{to} \py{not} \py{in} \self.updated
\kw{Transition-Output}:
    \py{return} \self.dist
\kw{Transition-Effect}:
    \self.updated.append(\kw{to})

\kw{Transition-Name}:
    receive(m)
\kw{Transition-Effect}:
    act = m + \self.weights[\kw{fr}]
    \py{if} \self.dist \py{is} \pyval{None} \py{or} act < \self.dist:
        \self.dist = act
	\self.parent = \kw{fr}
        \self.updated = []

\kw{Tasks}:
    send(m)
\endgroup
\end{framed}
\caption{Simio description of Bellman-Ford}
\end{figure}

We note that the code reads very similar to a formal I/O automaton
specification, which is one of Simio's goals. Simio now takes such an automaton
specification and parses it into an executable Python module, which is then
used by the simulator to run multiple nodes connected in a network.

\section{Simulation}

Cody.

\section{Related work}

{\color{red}Progress marker}

Of particular note in terms of related work are the
\href{http://groups.csail.mit.edu/tds/ioa/}{IOA Language and Toolset}, which
simulates the behaviour of a single I/O automata, and the
\href{http://www.veromodo.com/}{Tempo Toolset}, which simulates a single
*timed* I/O automata. These tools provide the user with a domain-specific
language for describing automata, as well as tools for simulating the operation
of a single automata, doing invariant checking, verifying liveness and safety
properties, and generating code.

While these both provide authors of automata with powerful tools for verifying
the correctness of their algorithms, they are not particularly useful for
users who want to study the behaviour of an automata so as to better understand
the underlying algorithm. In particular, we envision that students who are
being introduced to reasoning about distributed systems using I/O automata for
the first time, or seasoned researchers toying with an idea for a new
algorithm, would benefit greatly from being able to see how a network of nodes
behaves when running a specific algorithm. Furthermore, we believe that, by
allowing the user to specify the network graph they are interested in, they can
quickly explore corner cases that they suspect might be difficult for the
algorithm, or they can improve their understanding of the algorithm by
observing its behaviour in very different network topologies.

Another reason why we feel existing solutions are insufficient is that they
require the user to learn a new programming language simply to be able to type
up an algorithm. Instead, we propose to provide the user with the ability to
define their automata using the high-level programming language Python, which
many computer scientists are already familiar with. State variables will be
represented as Python variables, giving the user access to the suite of
primitive types and classes provided by Python and its many third-party
modules, without having to implement all these types ourselves. In the same
vein, transitions will be specified as pure Python functions whose arguments
are the current state variables, as well as any transition inputs, and whose
return value will be the new state of the object. Modeling transitions this way
allows us to give the full power of the Python language to the user, without
incurring additional implementation cost for us.

\section{Challenges}

The flexibility of expression in our system means that doing formal verification
or analysis of the automata becomes difficult. We surmise that this is the
reason why previous work generally chose to implement a language from scratch;
it gives them the opportunity to limit the language to only provide features
that can be reasoned about using current program analysis techniques. We
therefore position our simulator as a complement to existing tools, rather than
as a replacement for them. In particular, we anticipate that users will first
prototype their automata in our simulator to examine their behaviour, and then,
once the algorithm has been sufficiently well refined, write their final
automata specification using a specialized language such as Tempo so that
formal methods can be used to prove properties of the final algorithm.

\section{Project status}

At this stage in the project's development, we have built rudimentary support
for specifying automata, as well as a visualizing front-end that can display
nodes in a network, as well as show communication between nodes. While these
will both likely need further refinement, they are at the stage where we can
now focus on building a working automata simulator. This simulator would first
invoke the parser to get an executable version of the automata, construct and
initialize several copies of the automata to correspond to each of the nodes in
the simulated network, and link each automata as dictated by the network graph
specification. Once this setup has been performed, it would then start
executing the automata concurrently, ensuring fairness to each automata's task.
The information about messages being sent and nodes changing state would then
be communicated to the rendered in real-time, so that the visualization of the
running algorithm can be produced.

We anticipate that the biggest hurdle to making the simulator will be finding a
good way of handling non-determinism in the simulated automata while still
providing fairness. The reason for this is that a user may wish to bias the
non-determistic choices to force the algorithm to behave in a particular way.
It is not obvious how this can be done in a way that is both easy for the user,
and, at the same time, does not get in the way of the simulator ensuring that
the overall execution remains fair.
\end{document}
